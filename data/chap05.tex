\chapter{关键难点及解决措施}
\section{研究过程中可能遇到的关键难点}
根据前期的文献调研,在研究过程中可能会遇到以下关键难点:
\begin{enumerate}
	\item 理论模型和实际物理模型不相符。
	
	在建立连续型柔性臂的运动学模型的过程中,为了简化,往往认为臂段按照常曲率进行弯曲。然而,臂段的实际弯曲形态的物理规律是十分复杂的,而且是与机械臂的受力紧密相关。在实际模型中,绳索经过过孔时产生的摩擦力、臂段的重力、绳索对过孔的压力等都会造成臂段弯曲形态背离常曲率假设,因此,理论模型和物理模型之间很可能出现偏差。
	
	\item 实验系统的运动参数测量。
	
	由于柔性臂的臂段是不断弯曲变化的,因此在理论模型存在偏差的情况下,无法直接根据驱动空间参数得到臂段的弯曲角度、末端位置等运动参数。柔性臂结构紧凑,难以在臂段上直接安装传感器,因此直接对实验系统的运动参数进行测量可能是比较困难的。
	
\end{enumerate}
\section{相应的解决措施}
为解决以上问题,拟采用以下解决措施:
\begin{enumerate}
	\item 通过样机实验获得数据,进而采用模型辨识的方法对理论模型进行修正。
	
	虽然实际样机的准确模型会非常复杂,但驱动空间、构型空间和工作空间之间的映射关系应该是固定的,可以在理论模型的基础上,采集实验数据,采用拟合或神经网络等方法模型进行修正;
	
	\item 采用外部非接触式测量。
	
	为了获得柔性臂弯曲之后的形状参数,可以借助深度相机等视觉测量设备,实现非接触式测量。
\end{enumerate}
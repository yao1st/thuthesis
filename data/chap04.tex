\chapter{研究方案、预期成果及进度安排}

\section{研究方案与预期成果}
本毕业设计拟对绳索驱动超冗余连续型柔性臂的运动规划进行研究。首先建立连续型机器人的运动学模型,在此基础上探索其在复杂非结构化空间下的轨迹规划。通过研制物理样机,对运动学模型和逆解及规划算法进行验证。预期的成果有:
\begin{enumerate}
	\item 连续型机器人的位置级运动学模型。 
	
	结合物理样机,推导连续型机器人的完整运动学模型,建立驱动空间、构型空间和工作空间之间的映射关系,为后续的工作奠定基础;
	
	\item 球铰连接的连续型机械臂的微分运动学模型。
	
	借助旋量工具,推导具有球铰连接或柔性体连接结构的连续型机器人的微分运动学模型。建立Jacabian矩阵的构造方法,研究连续型机械臂的运动学奇异情况;
	
	\item 基于Jacobian矩阵伪逆的轨迹规划方法。
	
	利用Jacobian矩阵构造方法,研究基于伪逆的轨迹规划算法,通过仿真验证和样机实验进行验证;
	
	\item 基于模态函数的轨迹规划方法研究。
	
	采用脊线作为机器人宏观形状的刻画,提出基于脊线拟合的运动学逆解及规划算法,通过仿真实验和样机实验进行验证;
	
	\item 复杂空间中避障策略研究。
	
	在上述轨迹规划算法的基础上,研究连续型柔性臂在复杂非结构空间中的避障策略,使得其能够具备狭小空间中的作业能力,通过仿真实验和样机实验进行验证。
	
\end{enumerate}

\section{进度安排}
\begin{table}
	\centering
	\caption{毕业设计工作进度安排}
	\begin{tabular}{c|c}
		\toprule[1.5pt]
		时间 & 工作内容 \\
		\midrule		
		$\scriptsize{\sim}$2017.12.17 & 文献调研,物理样机的制作、装配和调试 \\
		2017.12.18 & 开题报告 \\
		2017.12.19 $\scriptsize{\sim}$ 2018.1.31 & 运动学模型的建立,运动学规划算法的仿真研究 \\
		2018.2.1 $\scriptsize{\sim}$ 2018.7.31 & 物理样机的进一步调试和实验 \\
		\bottomrule[1.5pt]
	\end{tabular}
\end{table}